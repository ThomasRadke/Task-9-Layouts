\documentclass{article}
\usepackage[margin=1in]{geometry}
\usepackage{amsmath, amssymb, amsthm, 
enumerate, 
enumitem,
fancyhdr, 
graphicx, 
hyperref, 
physics}

\usepackage[none]{hyphenat}

\hypersetup{colorlinks=true, linkcolor={}, urlcolor={blue}, citecolor={black}}
\urlstyle{same}

\usepackage{afterpage}

\newcommand\blankpage{
    \null
    \thispagestyle{empty}
    \addtocounter{page}{-1}
    \newpage
    }

%\setlength{\parindent}{0pt}

\linespread{2}

\pagestyle{fancy}

\lhead{\bf Linear Algebra}
\chead{\bf Final Portfolio}
\rhead{\bf Your Name}
\lfoot{}
\cfoot{\thepage}
\rfoot{}
\renewcommand{\headrulewidth}{0pt}
\renewcommand{\footrulewidth}{0pt}
\renewcommand*\footnoterule{}

\newtheorem{theorem}{Theorem}
\newtheorem{lemma}[theorem]{Lemma}
\newtheorem{definition}[theorem]{Definition}
\newtheorem{example}[theorem]{Example}

\newenvironment{solution}{\noindent{\textit{Solution.}}}{\hfill$\blacksquare$}

\newcommand{\ds}{\displaystyle}
\renewcommand{\l}{\left}
\renewcommand{\r}{\right}

\newcommand{\lv}{\l\lvert}
\newcommand{\rv}{\r\rvert}

\newcommand{\zbar}{\overline{z}}
\newcommand{\wbar}{\overline{w}}
\newcommand{\Arg}{\text{Arg}}

%%%%%%%%%%%%%%%%%%%%%%%%%%%%%%%%%%%%%%
%%%%%%%%%%%%%%%%%%%%%%%%%%%%%%%%%%%%%%

\begin{document}

\section{C.3 Extra Standard, similar to HW 6 Problem 2}
Type your response here to your first selected problem. Use ``sections'' to label each of your selected problems. Notice that \LaTeX \ automatically numbers each problem (or each section really). Your response to each problem should appear on its own page so when you print (single-sided), you can organize the pages in your portfolio as described in the formatting guidelines on the assignment sheet. 

\afterpage{\blankpage}	% Keep this here if you are unable to print one-sided.



\newpage

\section{E.3 Core Standard, similar to HW 9 Problem 4}
Type your response here to your second selected problem. 

\afterpage{\blankpage}	% Keep this here if you are unable to print one-sided.





\newpage

If needed, your references should be listed on their own page. See the last page for an example. There are two ways to build a bibliography in \LaTeX. For the purposes of this assignment, the most convenient way would be to use ``thebibliography" environment, for which you will need to manually order your references alphabetically by the author's last name and correctly format each ``bibitem'' using the Society for Industrial and Applied Mathematics (SIAM) citation style. For more details on the SIAM citaton style, see~\cite[References]{siam_cite} or go to the ``References'' section of
\begin{center}
\url{https://www.siam.org/Publications/Journals/About-SIAM-Journals/Information-for-Authors}
\end{center}
You may also find the SIAM Style Manual~\cite[pp. 88-93]{siam_style} useful. Be sure to compile twice so the reference numbers appear for your in-text citations.

The other way to build your bibliography is with BibTeX (use ``siam" as the bibliography style). For more information on these ways to build the bibliography, see
\begin{center}
\url{https://en.wikibooks.org/wiki/LaTeX/Bibliography_Management}
\end{center}

You should cite particular theorems, lemmas, and definitions. For example,
\begin{theorem}[{\cite[Theorem 2]{cowen06}}]
The statement of Theorem 2 from Cowen's journal article ``A new class of operators and a description of adjoints of composition operators'' should go here.
\end{theorem}
\begin{proof}
The proof of this theorem should go here.
\end{proof}

\begin{definition}[{\cite[p. 15]{bourdon08}}]
The definition on page 15 (if one exists) from Bourdon and Shapiro's journal article should go here. 
\end{definition}

\begin{example}
Find four branches of the multi-valued fourth root function that share the same branch cut.
\end{example}
\begin{solution}
The solution should go here.
\end{solution}

Let me know if you have questions.



\newpage

%%%%%%%%%%%%%%%%%%%%%%%%%
%     References  Page   %
%%%%%%%%%%%%%%%%%%%%%%%%%

\bibliographystyle{siam}
\begin{thebibliography}{99}

\bibitem{bourdon08}					%This is a journal article.
  \textsc{P. S. Bourdon and J. H. Shapiro},
  \emph{Adjoints of rationally induced composition operators},
  Journal of Functional Analysis, vol. 255 (2008), pp. 1995-2012, 
  \url{https://www.sciencedirect.com/science/article/pii/S0022123608002711}, (28 March 2017).

\bibitem{cowen06}					%This is another journal article.
  \textsc{C. C. Cowen and E. A. Gallardo-Guti\'{e}rrez},
  \emph{A new class of operators and a description of adjoints of composition operators},
  Journal of Functional Analysis, vol. 238 (2006), pp. 447-472,
  \url{https://www.sciencedirect.com/science/article/pii/S0022123606002588}, (28 March 2017).

\bibitem{cowen95}					%This is a book.
  \textsc{C. C. Cowen and B. D. MacCluer},
  \emph{Composition Operators on Spaces of Analytic Functions},
  CRC Press, Boca Raton,
  1995.
  
\bibitem{siam_cite}					%This is a page of a website.			
  \emph{Information for Authors},
  SIAM, 
  2019, 
  \url{https://www.siam.org/Publications/Journals/About-SIAM-Journals/Information-for-Authors}.
  
\bibitem{purdue_owl}					%This is a website.
  \emph{The Purdue OWL},
  Purdue U Writing Lab,
  2018,
  \url{https://owl.purdue.edu/owl/purdue_owl.html}.
  
\bibitem{siam_style}					%This is a chapter of a book.
  \emph{References},
  in SIAM Style Manual for Books and Journals,
  SIAM, Philadelphia, PA, 2013, pp. 88--96; also available online from
  \url{https://www.siam.org/Portals/0/Documents/stylemanual.pdf}
  
\end{thebibliography}

\end{document}
